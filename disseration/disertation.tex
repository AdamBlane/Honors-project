\documentclass{article}
\usepackage{graphicx}
\usepackage[utf8]{inputenc}
\usepackage[english]{babel}

\usepackage{hyperref}
\hypersetup{
	colorlinks=true,
	linkcolor=blue,
	filecolor=magenta,      
	urlcolor=cyan,
}

\begin{document}

\title{{}}
\author{Adam Blance}

\maketitle

\begin{abstract}
The abstract text goes here.
\end{abstract}

\section{Introduction}
From the beginning of the project the objective has been to develop a mathematical equation to evaluate escort mission maps. Then to procedurally generate maps that can successfully follow the evaluation technique created. The dissertation produced underneath documents the work that was involved in achieving this.
\subsection{Motivation}
With video games being one of the largest markets in the world right now, gaming companies must be developing and releasing their games as quickly as possible. If there is any possible way for them speed up releasing their games, they should take these approaches.
\vspace{5mm} 
\newline  
One of the biggest genres of games currently is team based shooters. With one of the larger games in the genre being Overwatch with currently over 35 million players worldwide. \href{https://twitter.com/PlayOverwatch/status/919925924769906688}. This genre of game has a very different development cycle to other games relying on continual updates adding new maps and characters to keep players interested and continually playing. To release new maps and characters on a regular basis can be difficult especially if the game has a large competitive scene because the new feature \vspace{5mm} must be fully tested before released.
\newline  
With testing new features in games becoming the most time consuming part of the development cycle, companies must look into other ways to streamline the testing of games. In this project one of the possible ways to decrease the time taken to test maps is explored. If companies can use a mathematical formula to test the fairness of maps, rather than the more conventional user based testing which can take up to several weeks, would allow for companies to produce more content in a quicker time frame, generating a greater profit and allowing for better content to be produced in the future. 


\subsection{Aims and Objectives}
There are two main aims of the project, the first is to create, implement and evaluate an algorithm that assesses escort mission maps once this has been complete, the next aim is to plan and develop a piece of software that can successfully follow the algorithm produced. The objectives below have been selected in order to accomplish these aims.
\begin{itemize}
	\item Research relevant topics and papers.
	\item Create or use an existing framework that can be used to develop the map.
	\item Design an algorithm.	
\end{itemize}
\subsection{Scope}
\subsubsection{Deliverable}
The items to be delivered from this project are: An algorithm that can evaluate any escort mission map and inform the user of the strengths and weakness of the map; A C++ program that can procedurally generate maps that can be tested by the algorithm; the results and analysis of the surveys that examine the credibility of the evaluation algorithm.  
\subsubsection{Boundaries}
When testing a map there are a large amount variables to be evaluated, as time is one of the limiting factors in this project it was decided that 3 major factors would be tested. They are:
\begin{itemize}
	\item The size of the map.
	\item The quality of the object that will be escorted's (payload) path.
	\item The overall balance"\"fairness of the map 	
\end{itemize}
\subsection{Constraints}
\subsection{Chapter Outlines}
\begin{itemize}
	\item Background - Discusses the research made into the topic.
	\item Methodology – States the methods used throughout the implementation process.
	\item Results – Displays the results from the tests carried out.
	\item Conclusion – Summarises results, assess s	
\end{itemize}
\section{Lit Riview/Background}

\section{Methodolgy}

\section{Results}

\section{Conclusion}
Write your conclusion here.

\bibliography{bib}
\bibliographystyle{plain}
\end{document}