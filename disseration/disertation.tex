\documentclass{article}
\usepackage{graphicx}
\usepackage[utf8]{inputenc}
\usepackage[english]{babel}

\usepackage{hyperref}
\hypersetup{
	colorlinks=true,
	linkcolor=blue,
	filecolor=magenta,      
	urlcolor=cyan,
}

\begin{document}

\title{{}}
\author{Adam Blance}

\maketitle

\begin{abstract}
The abstract text goes here.
\end{abstract}
\newpage
\renewcommand{\abstractname}{Acknowledgements}
\begin{abstract}
Shout out to my bois 
\end{abstract}
\newpage
\section{Introduction}
From the beginning of the project the objective has been to develop a mathematical equation to evaluate escort mission maps. Then to procedurally generate maps that can successfully follow the evaluation technique created. The dissertation produced underneath documents the work that was involved in achieving this.
\subsection{Motivation}
With video games being one of the largest markets in the world right now, gaming companies must be developing and releasing their games as quickly as possible. If there is any conceivable way for them speed up releasing their games, they should take these approaches.
\vspace{5mm} 
\newline  
One of the biggest genres of games currently is team based shooters. With one of the larger games in the genre being Overwatch with currently over 35 million players worldwide. \href{https://twitter.com/PlayOverwatch/status/919925924769906688}. This genre of game has a very different development cycle to other games relying on continual updates adding new maps and characters to keep players interested and continually playing. To release new maps and characters on a regular basis can be difficult especially if the game has a large competitive scene because the new feature \vspace{5mm} must be fully tested before released.
\newline  
With testing new features in games becoming the most time consuming part of the development cycle, companies must look into other ways to streamline the testing of games. In this project one of the possible ways to decrease the time taken to test maps is explored. If companies can use a mathematical formula to test the fairness of maps, rather than the more conventional user based testing which can take up to several weeks, would allow for companies to produce more content in a quicker time frame, generating a greater profit and allowing for better content to be produced in the future. 


\subsection{Aims and Objectives}
There are two main aims of the project, the first is to create, implement and evaluate an algorithm that assesses escort mission maps once this has been complete, the next aim is to plan and develop a piece of software that can successfully follow the algorithm produced. The objectives below have been selected in order to accomplish these aims.
\begin{itemize}
	\item Research relevant topics and papers.
	\item Create or use an existing framework that can be used to develop the map.
	\item Design an algorithm taking into account size, path of the payload and fairness. 
	\item Develop a programme that can procedurally generate escort mission maps.	
\end{itemize}
\subsection{Scope}
\subsubsection{Deliverable}
The items to be delivered from this project are: An algorithm that can evaluate any escort mission map and inform the user of the strengths and weakness of the map; A C++ program that can procedurally generate maps that can be tested by the algorithm; the results and analysis of the surveys that examine the credibility of the evaluation algorithm.  
\subsubsection{Boundaries and Constraints}
When testing a map there are a large amount variables to be evaluated, as time is one of the limiting factors in this project, it was decided that 3 major factors would be tested. They are:
\begin{itemize}
	\item The size of the map.
	\item The quality of the object that will be escorted's (payload) path.
	\item The overall balance\textbackslash fairness of the map 	
\end{itemize}
\subsection{Chapter Outlines}
\begin{itemize}
	\item Background - Discusses the research made into the topic.
	\item Methodology – States the methods used throughout the implementation process.
	\item Results – Displays the results from the tests carried out.
	\item Conclusion – Summarises results, assess s	
\end{itemize}
\newpage
\section{Background and Literature Review}
The
In this section the theory behind the honours project will be discussed and the areas of research that are used will be explained.
 
\subsection{Escort Missions}
An escort mission is a game-mode in team based shooters such as Overwatch and Loadout. In this mode there are two teams, the attacking and the defending team. The two teams have different aims in order to win. The attacking team attempts to escort the payload across the map to an objective while the defending team tries to stop them. While it is difficult to pin down the first game to have escort missions, one of the more notable uses was in 2007 with Team Fortress 2 by Valve, while it was not originally included in the game it was added in the first free update. It is important to note that the original Team Fortress had a game mode called Escort but this has many differences with its sequel such as the payload being a controllable player and there being three teams. Other games that use this game mode are Global Agenda and Wolfenstein: Enemy Territory.    
\subsubsection{Maps}
Escort mission maps are the play areas that escort missions are performed in. Between each game these maps can vary greatly so for this project, the definition of an escort mission map must be stated. The maps must have an attacking and defending spawn area. This is the place where the teams start at the beginning of the match and will return too whenever they die. The map must also have an unobstructed route that runs between the spawns, which the payload will be moved through. This route will be generated through pathfinding which is discussed below. 
\subsection{Path-Finding}
	
\subsubsection{Dijkstra’s Algorithm}
 Published by Edsger W. Dijkstra in 1959, Dijkstra’s algorithm is a way to calculate the shortest path between nodes on a graph. Solving the single-source shortest path problem. Nodes on a graph can represent either a 2-dimensional grid or points in a 3-D space. The most common example of a real life application of Dijkstra’s algorithm is being used in geographical maps to plot the fastest route between two cities, where each city is a node and the roads are associated with a weight. 
 \newline
 The algorithm works by initially marking the distance from the starting point to ever other vertex on the graph. 
 --------------------------Add how it works---------------------------
\subsubsection{A Star Pathfinding (A*)}
The A* algorithm was first presented in 1968 by Peter E. Hart as an improvement on Nils Nilsson’s A1 algorithm. A1 was designed to be a heuristic approach to increase the speed of Dijkstra’s algorithm invented back in 1964. A* is a search based algorithm that attempts to calculate the most optimal route to a specified goal by travelling through a weighted graph. 
\newline
The algorithm works by searching the nodes surrounding the current node using heuristic that estimate the distance from the end node.  A* contains two lists which are used to sort the nodes, open and closed. The open list contains all the nodes yet to be evaluated, while the closed list contains the nodes that have been assessed. Each node is given their own cost which is used to find the best path with the smaller cost the better. The score is calculated using the formula:
 \[f(n) = g(n) + h(n)\]  
 Where G is cost of travelling to the node and the h cost also known as the Heuristic cost is the cost to travel to the goal. This can be calculated in several ways and determines the efficiency and effectiveness of the algorithm. The F cost is the total cost of adding the two values together.
 \newline
 The total cost is used by A* to guide the algorithm towards the goal. At each iteration of the algorithm the node with the lowest total cost is removed from the open list and its neighbours are evaluated. This involves searching the closed list to see if the neighbour has already been evaluated and checking if the node cannot be passed. If either of these cases are true then ignore the node, otherwise the node is checked to see if it exists in the open list, if it isn’t its F cost is calculated and the node is added to the open list. If the node already exists in the open list then its G cost is compared with the current lowest G cost, if the new G cost is lower than this is the better path. This process progresses until the goal has been evaluated and added to the open list or all nodes have been evaluated and the open list is empty, if this happens the goal cannot be reached from the start point. 
 \newline
 
\subsection{Exsiting Work}
\section{Methodolgy}
\subsection{Introduction}
This section looks at the methods used during the implementation of the project. 
\section{Results}

\section{Conclusion}


\bibliographystyle{plain}
\bibliography{bib}

\end{document}